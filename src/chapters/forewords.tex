\chapter*{Kata Pengantar}
\addcontentsline{toc}{chapter}{KATA PENGANTAR}

%Gunakan bagian ini untuk memberikan ucapan terima kasih kepada semua pihak yang secara langsung atau tidak langsung membantu penyelesaian tugas akhir, termasuk pemberi beasiswa jika ada. Utamakan untuk memberikan ucapan terima kasih kepada tim pembimbing tugas akhir dan staf pengajar atau pihak program studi, bahkan sebelum mengucapkan terima kasih kepada keluarga. Ucapan terima kasih sebaiknya bukan hanya menyebutkan nama orang saja, tetapi juga memberikan penjelasan bagaimana bentuk bantuan/dukungan yang diberikan. Gunakan bahasa yang baik dan sopan serta memberikan kesan yang enak untuk dibaca. Sebagai contoh: “Tidak lupa saya ucapkan terima kasih kepada teman dekat saya, Tito, yang sejak satu tahun terakhir ini selalu memberikan semangat dan mengingatkan saya apabila lengah dalam mengerjakan Tugas Akhir ini. Tito juga banyak membantu mengoreksi format dan layout tulisan. Apresiasi saya sampaikan kepada pemberi beasiswa, Yayasan Beasiswa, yang telah memberikan bantuan dana kuliah dan biaya hidup selama dua tahun. Bantuan dana tersebut sangat membantu saya untuk dapat lebih fokus dalam menyelesaikan pendidikan saya. ....”. Ucapan permintaan maaf karena kekurangsempurnaan hasil Tugas Akhir tidak perlu ditulis.

Puji syukur kepada Tuhan Yang Maha Esa atas berkat dan karunia-Nya yang telah
memberikan kesempatan penulis untuk menyelesaikan salah satu kewajiban dalam
menempuh studi sarjana S1 pada Program Studi Teknik Telekomunikasi di Institut
Teknologi Bandung yaitu Tugas Akhir berjudul “\textbf{\thetitle}”.

Ucapan terima kasih dan rasa syukur juga tidak lupa disampaikan oleh penulis
kepada seluruh orang yang telah melancarkan dan membantu dalam pelaksanaan
Tugas Akhir yang telah diberikan baik dalam bentuk usaha, waktu, material dan
juga dukungan. Tanpa ada dukungan dari orang-orang tersebut, penulis tidak akan
mampu untuk menyelesaikan pengerjaan Tugas Akhir ini dengan baik. Maka
izinkanlah penulis menyampaikan rasa terima kasih kepada

\begin{enumerate}
    \item Orang Tua penulis yang selalu memberikan dukungan finansial maupun secara moral
    \item Prof. Ir. Hendrawan, M.Sc., Ph.D. selaku dosen pembimbing yang selalu membimbing dan membantu dalam pengerjaan tugas akhir ini
    \item Prof. Junchen Jiang dan Prof. Haryadi S. Gunawi selaku kolaborator yang selalu memberikan masukan
    \item Roy Huang selaku rekan kolaborator riset yang telah banyak membantu terkait hal teknis 
    \item Farhan Krishna selaku rekan TA yang selalu menjadi teman diskusi penulis
\end{enumerate}

Penulisan buku tugas akhir ini tidak akan bisa dilakukan tanpa adanya orang-orang
yang selalu membantu dalam penyelesaiannya. Penulis buku akhir in hanyalah
manusia yang tidak lepas dari kesalahan. Maka dari itu, penulis terbuka dan
menerima kritik, saran dan diskusi sebagai bahan perbaikan dan pembelajaran agar
penulis dapat menjadi pribadi yang lebih baik lagi kedepannya. Semoga Buku tugas
akhir yang penulis but mampu bermanfaat bagi pembaca, terutama teman-teman
pegiat telekomunikasi.

% \\[\baselineskip]
Bandung, \thedate \\[\baselineskip]
Penulis