\chapter{PENDAHULUAN}
% Hack: gatau kenapa harus gini
\pagenumbering{arabic}
\setcounter{page}{1}

% Bab Pendahuluan secara umum yang dijadikan landasan kerja dan arah kerja penulis tugas akhir, berfungsi mengantar pembaca untuk membaca laporan tugas akhir secara keseluruhan.

\section{Latar Belakang}

% Latar Belakang berisi dasar pemikiran, kebutuhan atau alasan yang menjadi ide dari topik tugas akhir. Tujuan utamanya adalah untuk memberikan informasi secukupnya kepada pembaca agar memahami topik yang akan dibahas.  Saat menuliskan bagian ini, posisikan anda sebagai pembaca – apakah anda tertarik untuk terus membaca?
Pada 5 tahun terakhir, perkembangan teknologi internet merupakan salah satu yang paling masif.
Hal ini dibuktikan dengan kehadiran standar komunikasi mobile 5G pada tahun 2019. 5G tidak hanya meningkatkan kecepatan internet menjadi
10Gb/s dari yang awalnya hanya 300Mb/s pada 4G LTE-Advanced \citep{4g}, namun 5G memiliki 2 \textit{use-case} lainnya
yakni \textit{Massive Machine-Type Communication} dan \textit{Ultra-Reliable Low-Latency Communication} \citep{5g}. 


% Security
Hadirnya \textit{Massive Machine-Type Communication} memungkinkan banyak perangkat yang dapat terkoneksi dalam sebuah jaringan.
Hal ini dapat mendukung keberlangsungan sistem IoT atau \textit{Internet of Things}. IoT adalah sebuah teknologi komunikasi
yang memungkinkan perangkat-perangkat dalam sebuah jaringan dapat berkomunikasi satu sama lainnya sehingga menciptakan sebuah sistem cerdas.
Perangkat-perangkat yang dimaksud dalam kasus ini adalah berbagai macam sensor. Data yang dikumpulkan dari berbagai macam sensor tersebut akan diproses
oleh sebuah perangkat komputasi terpusat yang saat ini dikenal dengan istilah \textit{cloud computing}

% Edge Computing
Cloud computing memiliki beberapa keterbatasan yakni lokasi yang terbatas sehingga memiliki waktu processing yang cukup lama
dan tidak cocok diterapkan pada sistem yang real-time. Untuk mengatasi masalah ini, lahirlah sebuah istilah baru, yakni Edge Computing.
Edge Computing adalah sebuah paradigma dalam cloud computing yang bertujuan untuk mendekatkan proses komputasi pada \textit{edge device}
sehingga pemrosesan dapat dilakukan dalam waktu singkat.

Masalah yang sering ditemukan pada Edge Computing adalah ketersediaan bandwidth terutama pada perangkat yang memiliki ukuran data yang relatif besar seperti
kamera keamanan. Kamera akan mengirimkan video pada server untuk dilakukan proses deteksi objek (kendaraan, orang, objek lainnya). Untuk mengatasi
masalah keterbatasan bandwidth, maka diperlukan lah sebuah sistem cerdas yang dapat mengalokasikan bandwidth secara periodik yang bergantung pada kebutuhan tiap kamera sehingga
didapatkan hasil akurasi yang maksimal.

Dari perumusan masalah tersebut, pada tugas akhir ini akan dirancang sebuah sistem \textit{resource allocator} yang berbasis
sensitivitas sehingga dapat meminimalkan trade-off antara bandwidth dan akurasi pada tiap kamera. 

% internet -> security issue -> IoT -> edge computing

\section{Rumusan Masalah}

% Rumusan Masalah berisi masalah utama yang dibahas dalam tugas akhir. Rumusan masalah yang baik memiliki struktur sebagai berikut:

% \begin{enumerate}
%     \item Penjelasan ringkas tentang kondisi/situasi yang ada sekarang terkait dengan topik utama yang dibahas tugas akhir.
%     \item Pokok persoalan dari kondisi/situasi yang ada, dapat dilihat dari kelemahan atau kekurangannya. Bagian ini merupakan inti dari rumusan masalah.
%     \item Elaborasi lebih lanjut yang menekankan pentingnya untuk menyelesaikan pokok persoalan tersebut.
%     \item Usulan singkat terkait dengan solusi yang ditawarkan untuk menyelesaikan persoalan.
% \end{enumerate}

% Penting untuk diperhatikan bahwa persoalan yang dideskripsikan pada subbab ini akan dipertanggungjawabkan di bab Evaluasi apakah terselesaikan atau tidak.

Masalah penelitian yang dirumuskan adalah apakah penggunaan \textit{sensitivity-based resource allocator} yang diajukan dapat meningkatkan 
rata-rata akurasi dari sistem \textit{video analytics} seperti DDS dan lebih baik dari videostorm?
\section{Tujuan dan Manfaat}

%Tuliskan tujuan utama dan/atau tujuan detil yang akan dicapai dalam pelaksanaan tugas akhir. Fokuskan pada hasil akhir yang ingin diperoleh setelah tugas akhir diselesaikan, terkait dengan penyelesaian persoalan pada rumusan masalah. Penting untuk diperhatikan bahwa tujuan yang dideskripsikan pada subbab ini akan dipertanggungjawabkan di akhir pelaksanaan tugas akhir apakah tercapai atau tidak.
Tujuan tugas akhir ini adalah untuk menghasilkan sebuah \textit{sensitivity-based resource allocator} yang dapat meningkatkan
rata-rata akurasi dari sistem \textit{video analytics}

\section{Lingkup Permasalahan}

%Tuliskan batasan-batasan yang diambil dalam pelaksanaan tugas akhir. Batasan ini dapat dihindari (tidak perlu ada) jika topik/judul tugas akhir dibuat cukup spesifik.
Pada topik \textit{Enhancing Video Analytics Accuracy via
Real-time Automated Video Compression Parameter Tuning as
well as Enabling Resource Allocator}, secara garis besar terdiri dari 2 buah subsistem, yakni \textit{video analytics application} dan \textit{resource allocator}.
Fokus penulis adalah untuk mengembangkan dan menguji \textit{resource allocator}. 
Selain itu, karena keterbatasan waktu yang diberikan dan alasan keamanan,
\textit{resource allocator} hanya dapat bekerja dalam jaringan lokal.


\section{Asumsi-asumsi}

Dalam pengerjaan tugas akhir ini terdapat asumsi-asumsi yang digunakan antara lain:

\begin{enumerate}
	\item Protokol komunikasi yang digunakan adalah HTTP
	\item Kondisi jaringan yang stabil sehingga komunikasi data dapat terus terjalin
	\item \textit{Resource allocator} hanya dapat dijalankan pada sistem operasi berbasis kernel Linux
	\item Jumlah \textit{client} atau kamera yang dapat terkoneksi adalah sebanyak 3 buah
	\item Lorem ipsum
\end{enumerate}

\section{Hipotesis}

Hipotesis adalah dugaan yang menjawab masalah penelitian. Hipotesis biasanya berupa pekerjaan atau teknik yang digunakan untuk mengatasi masalah penelitian. 
Hipotesis pekerjaan tugas akhir ini adalah:

\textit{Penggunaan perangkat Arduino yang mengntegrasikan data dari 4 buah sensor, transceiver jarak jauh LoRa, serta sumber energi hardware yang dapat melakukan self-recharging melalui solar panel, akan menghasilkan alat komunikasi ramah lingkungan untuk pengiriman data 4 buah sensor dari daerah yang belum terjangkau jaringan internet.}

Penggunaan \textit{resource allocator} yang diintegrasikan dengan gRPC dan \textit{self-adaptive VAP} dapat memberikan gambaran kondisi kebutuhan bandwidth
tiap kamera sehingga alokasi bandwidth ideal akan diperoleh yang berakibat pada peningkatan performansi (akurasi)

\section{Sistematika Buku Tugas Akhir}

% Template

% Subbab ini berisi jadwal pelaksanaan tugas akhir dan anggaran pelaksanaan tugas akhir. Contoh jadwal pelaksanaan tugas akhir ditunjukkan pada Gambar \ref{figure:contoh_jadwal_pelaksanaan} dan contoh anggaran pelaksanaan tugas akhir dirangkum dalam Tabel \ref{table:contoh_anggaran}.

% \begin{figure}[h]
% 	\small
% 	\centering
% 	\begin{ganttchart}[
% 		hgrid,
% 		vgrid,
% 		y unit chart=0.5cm,
% 		y unit title=0.6cm,
% 		title height=1,
% 		x unit=1mm,
% 		time slot format=isodate,
% 		time slot unit=day]{2020-09-01}{2020-12-31}
% 		\gantttitlecalendar{year, month, week=1} \\
% 		\ganttgroup{Grup Aktivitas 1}{2020-09-01}{2020-09-30} \\
% 		\ganttbar{Aktivitas 1}{2020-09-01}{2020-09-07} \\
% 		\ganttbar{Aktivitas 2}{2020-09-08}{2020-09-30} \\
% 		\ganttgroup{Grup Aktivitas 2}{2020-09-21}{2020-12-31} \\
% 		\ganttbar{Aktivitas 3}{2020-09-21}{2020-10-07} \\
% 		\ganttbar{Aktivitas 4}{2020-10-15}{2020-11-07} \\
% 		\ganttbar{Aktivitas 5}{2020-11-15}{2020-12-31} \\
% 		\ganttgroup{Grup Aktivitas 3}{2020-10-01}{2020-12-31}
% 	\end{ganttchart}
% 	\caption{Contoh Jadwal Pelaksanaan Tugas Akhir}
% 	\label{figure:contoh_jadwal_pelaksanaan}
% \end{figure}

% \begin{table}[htbp]
% 	\small
% 	\centering
% 	\caption{Anggaran Biaya Pelaksanaan Tugas Akhir}
% 	\label{table:contoh_anggaran}
% 	\begin{tabular}{lcrr}
% 		\toprule
% 		\multicolumn{1}{l}{\textbf{Hal}} & \multicolumn{1}{l}{\textbf{Satuan}} & \multicolumn{1}{l}{\textbf{Harga Satuan}} & \multicolumn{1}{r}{\textbf{Jumlah}}\\
% 		\midrule
% 		\textbf{Grup Keperluan 1} \\
% 		Keperluan 1 & 1 buah & Rp1.000.000,00 & Rp1.000.000,00 \\
% 		Keperluan 2 & 1 set & Rp400.000,00 & Rp400.000,00 \\
% 		\midrule
% 		\textbf{Grup Keperluan 2} \\
% 		Keperluan 3 & 1 buah & Rp2.000.000,00 & Rp2.000.000,00 \\
% 		Keperluan 4 & 2 buah & Rp300.000,00 & Rp600.000,00 \\
% 		\midrule
% 		\textbf{Total} & & & Rp4.000.000,00 \\
% 		\bottomrule
% 	\end{tabular}
% \end{table}
